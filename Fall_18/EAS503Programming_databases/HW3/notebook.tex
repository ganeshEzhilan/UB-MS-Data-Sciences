
% Default to the notebook output style

    


% Inherit from the specified cell style.




    
\documentclass[11pt]{article}

    
    
    \usepackage[T1]{fontenc}
    % Nicer default font (+ math font) than Computer Modern for most use cases
    \usepackage{mathpazo}

    % Basic figure setup, for now with no caption control since it's done
    % automatically by Pandoc (which extracts ![](path) syntax from Markdown).
    \usepackage{graphicx}
    % We will generate all images so they have a width \maxwidth. This means
    % that they will get their normal width if they fit onto the page, but
    % are scaled down if they would overflow the margins.
    \makeatletter
    \def\maxwidth{\ifdim\Gin@nat@width>\linewidth\linewidth
    \else\Gin@nat@width\fi}
    \makeatother
    \let\Oldincludegraphics\includegraphics
    % Set max figure width to be 80% of text width, for now hardcoded.
    \renewcommand{\includegraphics}[1]{\Oldincludegraphics[width=.8\maxwidth]{#1}}
    % Ensure that by default, figures have no caption (until we provide a
    % proper Figure object with a Caption API and a way to capture that
    % in the conversion process - todo).
    \usepackage{caption}
    \DeclareCaptionLabelFormat{nolabel}{}
    \captionsetup{labelformat=nolabel}

    \usepackage{adjustbox} % Used to constrain images to a maximum size 
    \usepackage{xcolor} % Allow colors to be defined
    \usepackage{enumerate} % Needed for markdown enumerations to work
    \usepackage{geometry} % Used to adjust the document margins
    \usepackage{amsmath} % Equations
    \usepackage{amssymb} % Equations
    \usepackage{textcomp} % defines textquotesingle
    % Hack from http://tex.stackexchange.com/a/47451/13684:
    \AtBeginDocument{%
        \def\PYZsq{\textquotesingle}% Upright quotes in Pygmentized code
    }
    \usepackage{upquote} % Upright quotes for verbatim code
    \usepackage{eurosym} % defines \euro
    \usepackage[mathletters]{ucs} % Extended unicode (utf-8) support
    \usepackage[utf8x]{inputenc} % Allow utf-8 characters in the tex document
    \usepackage{fancyvrb} % verbatim replacement that allows latex
    \usepackage{grffile} % extends the file name processing of package graphics 
                         % to support a larger range 
    % The hyperref package gives us a pdf with properly built
    % internal navigation ('pdf bookmarks' for the table of contents,
    % internal cross-reference links, web links for URLs, etc.)
    \usepackage{hyperref}
    \usepackage{longtable} % longtable support required by pandoc >1.10
    \usepackage{booktabs}  % table support for pandoc > 1.12.2
    \usepackage[inline]{enumitem} % IRkernel/repr support (it uses the enumerate* environment)
    \usepackage[normalem]{ulem} % ulem is needed to support strikethroughs (\sout)
                                % normalem makes italics be italics, not underlines
    

    
    
    % Colors for the hyperref package
    \definecolor{urlcolor}{rgb}{0,.145,.698}
    \definecolor{linkcolor}{rgb}{.71,0.21,0.01}
    \definecolor{citecolor}{rgb}{.12,.54,.11}

    % ANSI colors
    \definecolor{ansi-black}{HTML}{3E424D}
    \definecolor{ansi-black-intense}{HTML}{282C36}
    \definecolor{ansi-red}{HTML}{E75C58}
    \definecolor{ansi-red-intense}{HTML}{B22B31}
    \definecolor{ansi-green}{HTML}{00A250}
    \definecolor{ansi-green-intense}{HTML}{007427}
    \definecolor{ansi-yellow}{HTML}{DDB62B}
    \definecolor{ansi-yellow-intense}{HTML}{B27D12}
    \definecolor{ansi-blue}{HTML}{208FFB}
    \definecolor{ansi-blue-intense}{HTML}{0065CA}
    \definecolor{ansi-magenta}{HTML}{D160C4}
    \definecolor{ansi-magenta-intense}{HTML}{A03196}
    \definecolor{ansi-cyan}{HTML}{60C6C8}
    \definecolor{ansi-cyan-intense}{HTML}{258F8F}
    \definecolor{ansi-white}{HTML}{C5C1B4}
    \definecolor{ansi-white-intense}{HTML}{A1A6B2}

    % commands and environments needed by pandoc snippets
    % extracted from the output of `pandoc -s`
    \providecommand{\tightlist}{%
      \setlength{\itemsep}{0pt}\setlength{\parskip}{0pt}}
    \DefineVerbatimEnvironment{Highlighting}{Verbatim}{commandchars=\\\{\}}
    % Add ',fontsize=\small' for more characters per line
    \newenvironment{Shaded}{}{}
    \newcommand{\KeywordTok}[1]{\textcolor[rgb]{0.00,0.44,0.13}{\textbf{{#1}}}}
    \newcommand{\DataTypeTok}[1]{\textcolor[rgb]{0.56,0.13,0.00}{{#1}}}
    \newcommand{\DecValTok}[1]{\textcolor[rgb]{0.25,0.63,0.44}{{#1}}}
    \newcommand{\BaseNTok}[1]{\textcolor[rgb]{0.25,0.63,0.44}{{#1}}}
    \newcommand{\FloatTok}[1]{\textcolor[rgb]{0.25,0.63,0.44}{{#1}}}
    \newcommand{\CharTok}[1]{\textcolor[rgb]{0.25,0.44,0.63}{{#1}}}
    \newcommand{\StringTok}[1]{\textcolor[rgb]{0.25,0.44,0.63}{{#1}}}
    \newcommand{\CommentTok}[1]{\textcolor[rgb]{0.38,0.63,0.69}{\textit{{#1}}}}
    \newcommand{\OtherTok}[1]{\textcolor[rgb]{0.00,0.44,0.13}{{#1}}}
    \newcommand{\AlertTok}[1]{\textcolor[rgb]{1.00,0.00,0.00}{\textbf{{#1}}}}
    \newcommand{\FunctionTok}[1]{\textcolor[rgb]{0.02,0.16,0.49}{{#1}}}
    \newcommand{\RegionMarkerTok}[1]{{#1}}
    \newcommand{\ErrorTok}[1]{\textcolor[rgb]{1.00,0.00,0.00}{\textbf{{#1}}}}
    \newcommand{\NormalTok}[1]{{#1}}
    
    % Additional commands for more recent versions of Pandoc
    \newcommand{\ConstantTok}[1]{\textcolor[rgb]{0.53,0.00,0.00}{{#1}}}
    \newcommand{\SpecialCharTok}[1]{\textcolor[rgb]{0.25,0.44,0.63}{{#1}}}
    \newcommand{\VerbatimStringTok}[1]{\textcolor[rgb]{0.25,0.44,0.63}{{#1}}}
    \newcommand{\SpecialStringTok}[1]{\textcolor[rgb]{0.73,0.40,0.53}{{#1}}}
    \newcommand{\ImportTok}[1]{{#1}}
    \newcommand{\DocumentationTok}[1]{\textcolor[rgb]{0.73,0.13,0.13}{\textit{{#1}}}}
    \newcommand{\AnnotationTok}[1]{\textcolor[rgb]{0.38,0.63,0.69}{\textbf{\textit{{#1}}}}}
    \newcommand{\CommentVarTok}[1]{\textcolor[rgb]{0.38,0.63,0.69}{\textbf{\textit{{#1}}}}}
    \newcommand{\VariableTok}[1]{\textcolor[rgb]{0.10,0.09,0.49}{{#1}}}
    \newcommand{\ControlFlowTok}[1]{\textcolor[rgb]{0.00,0.44,0.13}{\textbf{{#1}}}}
    \newcommand{\OperatorTok}[1]{\textcolor[rgb]{0.40,0.40,0.40}{{#1}}}
    \newcommand{\BuiltInTok}[1]{{#1}}
    \newcommand{\ExtensionTok}[1]{{#1}}
    \newcommand{\PreprocessorTok}[1]{\textcolor[rgb]{0.74,0.48,0.00}{{#1}}}
    \newcommand{\AttributeTok}[1]{\textcolor[rgb]{0.49,0.56,0.16}{{#1}}}
    \newcommand{\InformationTok}[1]{\textcolor[rgb]{0.38,0.63,0.69}{\textbf{\textit{{#1}}}}}
    \newcommand{\WarningTok}[1]{\textcolor[rgb]{0.38,0.63,0.69}{\textbf{\textit{{#1}}}}}
    
    
    % Define a nice break command that doesn't care if a line doesn't already
    % exist.
    \def\br{\hspace*{\fill} \\* }
    % Math Jax compatability definitions
    \def\gt{>}
    \def\lt{<}
    % Document parameters
    \title{template\_eas503\_hw3}
    
    
    

    % Pygments definitions
    
\makeatletter
\def\PY@reset{\let\PY@it=\relax \let\PY@bf=\relax%
    \let\PY@ul=\relax \let\PY@tc=\relax%
    \let\PY@bc=\relax \let\PY@ff=\relax}
\def\PY@tok#1{\csname PY@tok@#1\endcsname}
\def\PY@toks#1+{\ifx\relax#1\empty\else%
    \PY@tok{#1}\expandafter\PY@toks\fi}
\def\PY@do#1{\PY@bc{\PY@tc{\PY@ul{%
    \PY@it{\PY@bf{\PY@ff{#1}}}}}}}
\def\PY#1#2{\PY@reset\PY@toks#1+\relax+\PY@do{#2}}

\expandafter\def\csname PY@tok@w\endcsname{\def\PY@tc##1{\textcolor[rgb]{0.73,0.73,0.73}{##1}}}
\expandafter\def\csname PY@tok@c\endcsname{\let\PY@it=\textit\def\PY@tc##1{\textcolor[rgb]{0.25,0.50,0.50}{##1}}}
\expandafter\def\csname PY@tok@cp\endcsname{\def\PY@tc##1{\textcolor[rgb]{0.74,0.48,0.00}{##1}}}
\expandafter\def\csname PY@tok@k\endcsname{\let\PY@bf=\textbf\def\PY@tc##1{\textcolor[rgb]{0.00,0.50,0.00}{##1}}}
\expandafter\def\csname PY@tok@kp\endcsname{\def\PY@tc##1{\textcolor[rgb]{0.00,0.50,0.00}{##1}}}
\expandafter\def\csname PY@tok@kt\endcsname{\def\PY@tc##1{\textcolor[rgb]{0.69,0.00,0.25}{##1}}}
\expandafter\def\csname PY@tok@o\endcsname{\def\PY@tc##1{\textcolor[rgb]{0.40,0.40,0.40}{##1}}}
\expandafter\def\csname PY@tok@ow\endcsname{\let\PY@bf=\textbf\def\PY@tc##1{\textcolor[rgb]{0.67,0.13,1.00}{##1}}}
\expandafter\def\csname PY@tok@nb\endcsname{\def\PY@tc##1{\textcolor[rgb]{0.00,0.50,0.00}{##1}}}
\expandafter\def\csname PY@tok@nf\endcsname{\def\PY@tc##1{\textcolor[rgb]{0.00,0.00,1.00}{##1}}}
\expandafter\def\csname PY@tok@nc\endcsname{\let\PY@bf=\textbf\def\PY@tc##1{\textcolor[rgb]{0.00,0.00,1.00}{##1}}}
\expandafter\def\csname PY@tok@nn\endcsname{\let\PY@bf=\textbf\def\PY@tc##1{\textcolor[rgb]{0.00,0.00,1.00}{##1}}}
\expandafter\def\csname PY@tok@ne\endcsname{\let\PY@bf=\textbf\def\PY@tc##1{\textcolor[rgb]{0.82,0.25,0.23}{##1}}}
\expandafter\def\csname PY@tok@nv\endcsname{\def\PY@tc##1{\textcolor[rgb]{0.10,0.09,0.49}{##1}}}
\expandafter\def\csname PY@tok@no\endcsname{\def\PY@tc##1{\textcolor[rgb]{0.53,0.00,0.00}{##1}}}
\expandafter\def\csname PY@tok@nl\endcsname{\def\PY@tc##1{\textcolor[rgb]{0.63,0.63,0.00}{##1}}}
\expandafter\def\csname PY@tok@ni\endcsname{\let\PY@bf=\textbf\def\PY@tc##1{\textcolor[rgb]{0.60,0.60,0.60}{##1}}}
\expandafter\def\csname PY@tok@na\endcsname{\def\PY@tc##1{\textcolor[rgb]{0.49,0.56,0.16}{##1}}}
\expandafter\def\csname PY@tok@nt\endcsname{\let\PY@bf=\textbf\def\PY@tc##1{\textcolor[rgb]{0.00,0.50,0.00}{##1}}}
\expandafter\def\csname PY@tok@nd\endcsname{\def\PY@tc##1{\textcolor[rgb]{0.67,0.13,1.00}{##1}}}
\expandafter\def\csname PY@tok@s\endcsname{\def\PY@tc##1{\textcolor[rgb]{0.73,0.13,0.13}{##1}}}
\expandafter\def\csname PY@tok@sd\endcsname{\let\PY@it=\textit\def\PY@tc##1{\textcolor[rgb]{0.73,0.13,0.13}{##1}}}
\expandafter\def\csname PY@tok@si\endcsname{\let\PY@bf=\textbf\def\PY@tc##1{\textcolor[rgb]{0.73,0.40,0.53}{##1}}}
\expandafter\def\csname PY@tok@se\endcsname{\let\PY@bf=\textbf\def\PY@tc##1{\textcolor[rgb]{0.73,0.40,0.13}{##1}}}
\expandafter\def\csname PY@tok@sr\endcsname{\def\PY@tc##1{\textcolor[rgb]{0.73,0.40,0.53}{##1}}}
\expandafter\def\csname PY@tok@ss\endcsname{\def\PY@tc##1{\textcolor[rgb]{0.10,0.09,0.49}{##1}}}
\expandafter\def\csname PY@tok@sx\endcsname{\def\PY@tc##1{\textcolor[rgb]{0.00,0.50,0.00}{##1}}}
\expandafter\def\csname PY@tok@m\endcsname{\def\PY@tc##1{\textcolor[rgb]{0.40,0.40,0.40}{##1}}}
\expandafter\def\csname PY@tok@gh\endcsname{\let\PY@bf=\textbf\def\PY@tc##1{\textcolor[rgb]{0.00,0.00,0.50}{##1}}}
\expandafter\def\csname PY@tok@gu\endcsname{\let\PY@bf=\textbf\def\PY@tc##1{\textcolor[rgb]{0.50,0.00,0.50}{##1}}}
\expandafter\def\csname PY@tok@gd\endcsname{\def\PY@tc##1{\textcolor[rgb]{0.63,0.00,0.00}{##1}}}
\expandafter\def\csname PY@tok@gi\endcsname{\def\PY@tc##1{\textcolor[rgb]{0.00,0.63,0.00}{##1}}}
\expandafter\def\csname PY@tok@gr\endcsname{\def\PY@tc##1{\textcolor[rgb]{1.00,0.00,0.00}{##1}}}
\expandafter\def\csname PY@tok@ge\endcsname{\let\PY@it=\textit}
\expandafter\def\csname PY@tok@gs\endcsname{\let\PY@bf=\textbf}
\expandafter\def\csname PY@tok@gp\endcsname{\let\PY@bf=\textbf\def\PY@tc##1{\textcolor[rgb]{0.00,0.00,0.50}{##1}}}
\expandafter\def\csname PY@tok@go\endcsname{\def\PY@tc##1{\textcolor[rgb]{0.53,0.53,0.53}{##1}}}
\expandafter\def\csname PY@tok@gt\endcsname{\def\PY@tc##1{\textcolor[rgb]{0.00,0.27,0.87}{##1}}}
\expandafter\def\csname PY@tok@err\endcsname{\def\PY@bc##1{\setlength{\fboxsep}{0pt}\fcolorbox[rgb]{1.00,0.00,0.00}{1,1,1}{\strut ##1}}}
\expandafter\def\csname PY@tok@kc\endcsname{\let\PY@bf=\textbf\def\PY@tc##1{\textcolor[rgb]{0.00,0.50,0.00}{##1}}}
\expandafter\def\csname PY@tok@kd\endcsname{\let\PY@bf=\textbf\def\PY@tc##1{\textcolor[rgb]{0.00,0.50,0.00}{##1}}}
\expandafter\def\csname PY@tok@kn\endcsname{\let\PY@bf=\textbf\def\PY@tc##1{\textcolor[rgb]{0.00,0.50,0.00}{##1}}}
\expandafter\def\csname PY@tok@kr\endcsname{\let\PY@bf=\textbf\def\PY@tc##1{\textcolor[rgb]{0.00,0.50,0.00}{##1}}}
\expandafter\def\csname PY@tok@bp\endcsname{\def\PY@tc##1{\textcolor[rgb]{0.00,0.50,0.00}{##1}}}
\expandafter\def\csname PY@tok@fm\endcsname{\def\PY@tc##1{\textcolor[rgb]{0.00,0.00,1.00}{##1}}}
\expandafter\def\csname PY@tok@vc\endcsname{\def\PY@tc##1{\textcolor[rgb]{0.10,0.09,0.49}{##1}}}
\expandafter\def\csname PY@tok@vg\endcsname{\def\PY@tc##1{\textcolor[rgb]{0.10,0.09,0.49}{##1}}}
\expandafter\def\csname PY@tok@vi\endcsname{\def\PY@tc##1{\textcolor[rgb]{0.10,0.09,0.49}{##1}}}
\expandafter\def\csname PY@tok@vm\endcsname{\def\PY@tc##1{\textcolor[rgb]{0.10,0.09,0.49}{##1}}}
\expandafter\def\csname PY@tok@sa\endcsname{\def\PY@tc##1{\textcolor[rgb]{0.73,0.13,0.13}{##1}}}
\expandafter\def\csname PY@tok@sb\endcsname{\def\PY@tc##1{\textcolor[rgb]{0.73,0.13,0.13}{##1}}}
\expandafter\def\csname PY@tok@sc\endcsname{\def\PY@tc##1{\textcolor[rgb]{0.73,0.13,0.13}{##1}}}
\expandafter\def\csname PY@tok@dl\endcsname{\def\PY@tc##1{\textcolor[rgb]{0.73,0.13,0.13}{##1}}}
\expandafter\def\csname PY@tok@s2\endcsname{\def\PY@tc##1{\textcolor[rgb]{0.73,0.13,0.13}{##1}}}
\expandafter\def\csname PY@tok@sh\endcsname{\def\PY@tc##1{\textcolor[rgb]{0.73,0.13,0.13}{##1}}}
\expandafter\def\csname PY@tok@s1\endcsname{\def\PY@tc##1{\textcolor[rgb]{0.73,0.13,0.13}{##1}}}
\expandafter\def\csname PY@tok@mb\endcsname{\def\PY@tc##1{\textcolor[rgb]{0.40,0.40,0.40}{##1}}}
\expandafter\def\csname PY@tok@mf\endcsname{\def\PY@tc##1{\textcolor[rgb]{0.40,0.40,0.40}{##1}}}
\expandafter\def\csname PY@tok@mh\endcsname{\def\PY@tc##1{\textcolor[rgb]{0.40,0.40,0.40}{##1}}}
\expandafter\def\csname PY@tok@mi\endcsname{\def\PY@tc##1{\textcolor[rgb]{0.40,0.40,0.40}{##1}}}
\expandafter\def\csname PY@tok@il\endcsname{\def\PY@tc##1{\textcolor[rgb]{0.40,0.40,0.40}{##1}}}
\expandafter\def\csname PY@tok@mo\endcsname{\def\PY@tc##1{\textcolor[rgb]{0.40,0.40,0.40}{##1}}}
\expandafter\def\csname PY@tok@ch\endcsname{\let\PY@it=\textit\def\PY@tc##1{\textcolor[rgb]{0.25,0.50,0.50}{##1}}}
\expandafter\def\csname PY@tok@cm\endcsname{\let\PY@it=\textit\def\PY@tc##1{\textcolor[rgb]{0.25,0.50,0.50}{##1}}}
\expandafter\def\csname PY@tok@cpf\endcsname{\let\PY@it=\textit\def\PY@tc##1{\textcolor[rgb]{0.25,0.50,0.50}{##1}}}
\expandafter\def\csname PY@tok@c1\endcsname{\let\PY@it=\textit\def\PY@tc##1{\textcolor[rgb]{0.25,0.50,0.50}{##1}}}
\expandafter\def\csname PY@tok@cs\endcsname{\let\PY@it=\textit\def\PY@tc##1{\textcolor[rgb]{0.25,0.50,0.50}{##1}}}

\def\PYZbs{\char`\\}
\def\PYZus{\char`\_}
\def\PYZob{\char`\{}
\def\PYZcb{\char`\}}
\def\PYZca{\char`\^}
\def\PYZam{\char`\&}
\def\PYZlt{\char`\<}
\def\PYZgt{\char`\>}
\def\PYZsh{\char`\#}
\def\PYZpc{\char`\%}
\def\PYZdl{\char`\$}
\def\PYZhy{\char`\-}
\def\PYZsq{\char`\'}
\def\PYZdq{\char`\"}
\def\PYZti{\char`\~}
% for compatibility with earlier versions
\def\PYZat{@}
\def\PYZlb{[}
\def\PYZrb{]}
\makeatother


    % Exact colors from NB
    \definecolor{incolor}{rgb}{0.0, 0.0, 0.5}
    \definecolor{outcolor}{rgb}{0.545, 0.0, 0.0}



    
    % Prevent overflowing lines due to hard-to-break entities
    \sloppy 
    % Setup hyperref package
    \hypersetup{
      breaklinks=true,  % so long urls are correctly broken across lines
      colorlinks=true,
      urlcolor=urlcolor,
      linkcolor=linkcolor,
      citecolor=citecolor,
      }
    % Slightly bigger margins than the latex defaults
    
    \geometry{verbose,tmargin=1in,bmargin=1in,lmargin=1in,rmargin=1in}
    
    

    \begin{document}
    
    
    \maketitle
    
    

    
    \subsection{EAS 503 Homework 3
Submission}\label{eas-503-homework-3-submission}

Name - \emph{Enter your name here}

**Make sure that you rename the notebook file to
\emph{replacewithubitname}\_eas503\_hw3.ipynb**

\subsubsection{Submission Details}\label{submission-details}

\begin{itemize}
\tightlist
\item
  \textbf{Due Date} - November 13, 2018 by 11:59 PM EST. All assignments
  have to be submitted using UBLearns.
\item
  \textbf{Number of Problems} - 3
\item
  \textbf{Maximum points} - 100
\item
  \textbf{Collaboration policy}
\item
  Every student has to submit individual homeworks
\item
  Any collaboration, in the form of discussion, with other members of
  the class is permitted, as long as the names of the collaborating
  members are explicitly stated on top of the submitted homework.
\item
  Any overlap with another submission or material from Internet will be
  awarded an F.
\item
  \textbf{Late submission policy} - \emph{No late submissions allowed}
\item
  \textbf{Submission Format} - Enter code for each problem in the
  appropriate cell below. You may use multiple cells for a single
  problem to improve readability.
\end{itemize}

    \subsubsection{\texorpdfstring{Problem 1 - Using regular expressions in
\texttt{Python} (50
points)}{Problem 1 - Using regular expressions in Python (50 points)}}\label{problem-1---using-regular-expressions-in-python-50-points}

In this problem you will write a rudimentary web crawler to extract
information from \texttt{Wikipedia}.

The three requirements for this problem are: \#\#\#\#\# 1. Get a
chronological list of US Presidents from the
\href{https://en.wikipedia.org/wiki/List_of_Presidents_of_the_United_States}{List
of Presidents of the United States} wikipedia entry (20 points).

You will have to write a function called \texttt{getPresidents(url)}
which takes one argument, a string containing the url, and returns the
list of URLs for the wikipedia pages for each US President in the
chronological order of their presidency, i.e.,
\texttt{{[}\textquotesingle{}https://en.wikipedia.org/wiki/George\_Washington\textquotesingle{},\ \textquotesingle{}https://en.wikipedia.org/wiki/John\_Adams\textquotesingle{}{]}\ ...}

The url argument to the function should be -
https://en.wikipedia.org/wiki/List\_of\_Presidents\_of\_the\_United\_States

You will notice that one president, Grover Cleveland, will appear twice.
For his case, remove the second entry. Reason he comes twice will be
apparent once you visit his entry on \texttt{Wikipedia}. While there you
will also notice that he was the Mayor of Buffalo once, which is why we
have the Grover Cleveland Highway in South Campus.

To write this function you will need to use the module \texttt{requests}
which allows grabbing the html from any url. For instance:

\begin{Shaded}
\begin{Highlighting}[]
\ImportTok{import}\NormalTok{ requests}
\ImportTok{import}\NormalTok{ re}

\NormalTok{url }\OperatorTok{=} \StringTok{"https://www.cse.buffalo.edu"}
\NormalTok{urlreq }\OperatorTok{=}\NormalTok{ request.get(url)}
\NormalTok{urltext }\OperatorTok{=}\NormalTok{ urlreq.txt}
\NormalTok{urllines }\OperatorTok{=}\NormalTok{ re.split(}\VerbatimStringTok{r'\textbackslash{}n'}\NormalTok{,urltext)}
\end{Highlighting}
\end{Shaded}

In the above snippet, urllines will be a \texttt{list} containing lines
of the html code in the above website.

You will then write a parser that will go through the html lines and
extract the list of presidents. You can check the actual page to see
where that list is.

\textbf{Note:} No points will be awarded to this problem if you have
manually created the list or derived the list from an alternative url.

\subparagraph{\texorpdfstring{2. Parse each President entry in
\texttt{Wikipedia} and extract his date of birth. (20
points)}{2. Parse each President entry in Wikipedia and extract his date of birth. (20 points)}}\label{parse-each-president-entry-in-wikipedia-and-extract-his-date-of-birth.-20-points}

On each President's wikipedia entry, the \texttt{infobox} on the right
lists, among other things, his exact birthday. Implement a function,
called \texttt{getBirthdays(urls)} which takes as input the list that is
returned by the above implemented \texttt{getPresidents(url)} function
and returns a \texttt{Pandas} \texttt{TimeSeries} object, that simply
contains the names of the Presidents as the value, and the birthdate as
the time index.

\subparagraph{\texorpdfstring{3. Plot an annotated \texttt{timeline}
plot using \texttt{matplotlib}. (10
points)}{3. Plot an annotated timeline plot using matplotlib. (10 points)}}\label{plot-an-annotated-timeline-plot-using-matplotlib.-10-points}

Use \texttt{matplotlib} routines to plot \textbf{time line} of the US
presidents, where the x-axis corresponds to time from 1720 until 2000
and y-axis denotes the birthdate for each president. The timeline plot
should look similar to:

\begin{figure}
\centering
\includegraphics{attachment:beatles_timeline.png}
\caption{beatles\_timeline.png}
\end{figure}

You will need to adjust the heights, font sizes, etc., to make it look
clean.

Note that there is not direct function for plotting a module, you will
just need to iterate over the values to create several overlaid plots
with desired effects.

\paragraph{Points breakup:}\label{points-breakup}

This problem is worth 50 points, and first two subparts are worth 20
points each. For the plotting subpart, a clean looking plot with all
information will fetch you 10 points. A cluttered plot, with all
information, will only fetch you 5 points.

    \begin{Verbatim}[commandchars=\\\{\}]
{\color{incolor}In [{\color{incolor} }]:} \PY{k}{def} \PY{n+nf}{getPresidents}\PY{p}{(}\PY{n}{url}\PY{p}{)}\PY{p}{:}
            \PY{c+c1}{\PYZsh{}\PYZsh{} add your code here}
\end{Verbatim}


    \begin{Verbatim}[commandchars=\\\{\}]
{\color{incolor}In [{\color{incolor} }]:} \PY{k}{def} \PY{n+nf}{getBirthdays}\PY{p}{(}\PY{n}{url}\PY{p}{)}\PY{p}{:}
            \PY{c+c1}{\PYZsh{}\PYZsh{} add your code here}
\end{Verbatim}


    \begin{Verbatim}[commandchars=\\\{\}]
{\color{incolor}In [{\color{incolor} }]:} \PY{c+c1}{\PYZsh{}\PYZsh{}\PYZsh{} add plotting code here}
\end{Verbatim}


    \subsubsection{Problem 2 - Chicago Crime Data (25
points)}\label{problem-2---chicago-crime-data-25-points}

The Chicago Crime Data reflects the crime incidents that occured in the
city of Chicago during 2016 and 2017. The data is provided in the
\texttt{chicago\_crime\_data\_v3.csv} file. For each crime incident,
there is information regarding the case ID, crime location, description,
primary crime type ID, coordinates of the crime etc. To identify the
primary crime type (denoted by the \texttt{Primary\ type} variable), the
\texttt{primary\_type\_ID.csv} file is provided which maps the primary
crime type to their corresponding ID.

For this problem, the goal is to identify the crime types that have
occurred more than 10,000 times during the years 2016 and 2017. For
this,

\begin{enumerate}
\def\labelenumi{\arabic{enumi}.}
\item
  Import the CSV files as dataframes using pandas.
\item
  Sort the crime data file by crime ID and merge the two files by
  \texttt{Primary\ Type\ ID} variable.
\item
  Identify the crime types that have a frequency \textgreater{}10,000
  and list them as \texttt{High\ Frequency\ Crime\ Types}. Provide a
  timeseries plot for all \texttt{High\ Frequency\ Crime\ Types}. For
  this, compute the number of crimes that occur per each crime type per
  month (2016 Jan - 2017 Dec). Use the \texttt{series.plot} method in
  Pandas for Series objects.
\item
  Using the code provided below, plot all the crimes that come under
  \texttt{High\ Frequency\ Crime\ Types} on a map of Chicago. For this,
  use the location information provided in the data. For any crimes with
  missing location coordinates, sort the data by ascending \texttt{Date}
  and use forward fill.
\end{enumerate}

(Make sure that each data point must be plotted as a transparent
circular dots with a markersize of atleast 20 and alpha set to atmost
0.5 (transparency))

\emph{Installation Notes}: You will need the \texttt{basemap} package.
You can get that using:

\begin{verbatim}
conda install basemap
\end{verbatim}

Due to some bug in the way \texttt{conda} maintains environmental
variables, the following import statement might not work:

\begin{Shaded}
\begin{Highlighting}[]
\ImportTok{from}\NormalTok{ mpl_toolkits.basemap }\ImportTok{import}\NormalTok{ Basemap}
\end{Highlighting}
\end{Shaded}

If it does not work on your computer, try the following:

\begin{Shaded}
\begin{Highlighting}[]
\ImportTok{import}\NormalTok{ os,conda}

\NormalTok{conda_file_dir }\OperatorTok{=}\NormalTok{ conda.}\VariableTok{__file__}
\NormalTok{conda_dir }\OperatorTok{=}\NormalTok{ conda_file_dir.split(}\StringTok{'lib'}\NormalTok{)[}\DecValTok{0}\NormalTok{]}
\NormalTok{proj_lib }\OperatorTok{=}\NormalTok{ os.path.join(os.path.join(conda_dir, }\StringTok{'share'}\NormalTok{), }\StringTok{'proj'}\NormalTok{)}
\NormalTok{os.environ[}\StringTok{"PROJ_LIB"}\NormalTok{] }\OperatorTok{=}\NormalTok{ proj_lib}

\ImportTok{from}\NormalTok{ mpl_toolkits.basemap }\ImportTok{import}\NormalTok{ Basemap}
\end{Highlighting}
\end{Shaded}

    \begin{Verbatim}[commandchars=\\\{\}]
{\color{incolor}In [{\color{incolor}3}]:} \PY{k+kn}{import} \PY{n+nn}{os}\PY{o}{,}\PY{n+nn}{conda}
        \PY{n}{conda\PYZus{}file\PYZus{}dir} \PY{o}{=} \PY{n}{conda}\PY{o}{.}\PY{n+nv+vm}{\PYZus{}\PYZus{}file\PYZus{}\PYZus{}}
        \PY{n}{conda\PYZus{}dir} \PY{o}{=} \PY{n}{conda\PYZus{}file\PYZus{}dir}\PY{o}{.}\PY{n}{split}\PY{p}{(}\PY{l+s+s1}{\PYZsq{}}\PY{l+s+s1}{lib}\PY{l+s+s1}{\PYZsq{}}\PY{p}{)}\PY{p}{[}\PY{l+m+mi}{0}\PY{p}{]}
        \PY{n}{proj\PYZus{}lib} \PY{o}{=} \PY{n}{os}\PY{o}{.}\PY{n}{path}\PY{o}{.}\PY{n}{join}\PY{p}{(}\PY{n}{os}\PY{o}{.}\PY{n}{path}\PY{o}{.}\PY{n}{join}\PY{p}{(}\PY{n}{conda\PYZus{}dir}\PY{p}{,} \PY{l+s+s1}{\PYZsq{}}\PY{l+s+s1}{share}\PY{l+s+s1}{\PYZsq{}}\PY{p}{)}\PY{p}{,} \PY{l+s+s1}{\PYZsq{}}\PY{l+s+s1}{proj}\PY{l+s+s1}{\PYZsq{}}\PY{p}{)}
        \PY{n}{os}\PY{o}{.}\PY{n}{environ}\PY{p}{[}\PY{l+s+s2}{\PYZdq{}}\PY{l+s+s2}{PROJ\PYZus{}LIB}\PY{l+s+s2}{\PYZdq{}}\PY{p}{]} \PY{o}{=} \PY{n}{proj\PYZus{}lib}
        
        \PY{k+kn}{from} \PY{n+nn}{mpl\PYZus{}toolkits}\PY{n+nn}{.}\PY{n+nn}{basemap} \PY{k}{import} \PY{n}{Basemap}
        \PY{k+kn}{import} \PY{n+nn}{numpy} \PY{k}{as} \PY{n+nn}{np}
        \PY{k+kn}{import} \PY{n+nn}{matplotlib}\PY{n+nn}{.}\PY{n+nn}{pyplot} \PY{k}{as} \PY{n+nn}{plt}
        
        \PY{n}{fig} \PY{o}{=} \PY{n}{plt}\PY{o}{.}\PY{n}{figure}\PY{p}{(}\PY{n}{figsize}\PY{o}{=}\PY{p}{[}\PY{l+m+mi}{16}\PY{p}{,}\PY{l+m+mi}{10}\PY{p}{]}\PY{p}{)}
        \PY{n}{m} \PY{o}{=} \PY{n}{Basemap}\PY{p}{(}\PY{n}{projection}\PY{o}{=}\PY{l+s+s1}{\PYZsq{}}\PY{l+s+s1}{merc}\PY{l+s+s1}{\PYZsq{}}\PY{p}{,}\PY{n}{llcrnrlat}\PY{o}{=}\PY{l+m+mf}{41.60}\PY{p}{,}\PY{n}{urcrnrlat}\PY{o}{=}\PY{l+m+mf}{42.10}\PY{p}{,}\PYZbs{}
                \PY{n}{llcrnrlon}\PY{o}{=}\PY{o}{\PYZhy{}}\PY{l+m+mf}{88.0}\PY{p}{,}\PY{n}{urcrnrlon}\PY{o}{=}\PY{o}{\PYZhy{}}\PY{l+m+mf}{87.50}\PY{p}{,}\PY{n}{lat\PYZus{}ts}\PY{o}{=}\PY{l+m+mi}{20}\PY{p}{,}\PY{n}{resolution}\PY{o}{=}\PY{l+s+s1}{\PYZsq{}}\PY{l+s+s1}{c}\PY{l+s+s1}{\PYZsq{}}\PY{p}{)}
        \PY{n}{m}\PY{o}{.}\PY{n}{readshapefile}\PY{p}{(}\PY{l+s+s1}{\PYZsq{}}\PY{l+s+s1}{./geo\PYZus{}export\PYZus{}0e3da441\PYZhy{}8fe8\PYZhy{}4e11\PYZhy{}9ca0\PYZhy{}42ef75cab68e}\PY{l+s+s1}{\PYZsq{}}\PY{p}{,}\PY{l+s+s1}{\PYZsq{}}\PY{l+s+s1}{chicago}\PY{l+s+s1}{\PYZsq{}}\PY{p}{)}
        \PY{n}{lons} \PY{o}{=} \PY{p}{[}\PY{o}{\PYZhy{}}\PY{l+m+mf}{87.6}\PY{p}{,}\PY{o}{\PYZhy{}}\PY{l+m+mf}{87.7}\PY{p}{]}
        \PY{n}{lats} \PY{o}{=} \PY{p}{[}\PY{l+m+mf}{41.65}\PY{p}{,}\PY{l+m+mf}{41.75}\PY{p}{]}
        \PY{n}{x}\PY{p}{,}\PY{n}{y} \PY{o}{=} \PY{n}{m}\PY{p}{(}\PY{n}{lons}\PY{p}{,}\PY{n}{lats}\PY{p}{)}
        \PY{n}{plt}\PY{o}{.}\PY{n}{scatter}\PY{p}{(}\PY{n}{x}\PY{p}{,}\PY{n}{y}\PY{p}{,}\PY{n}{alpha}\PY{o}{=}\PY{l+m+mf}{0.5}\PY{p}{,}\PY{n}{s}\PY{o}{=}\PY{l+m+mi}{20}\PY{p}{)}
        \PY{n}{plt}\PY{o}{.}\PY{n}{title}\PY{p}{(}\PY{l+s+s2}{\PYZdq{}}\PY{l+s+s2}{Chicago Cenus Tracts}\PY{l+s+s2}{\PYZdq{}}\PY{p}{)}
        \PY{n}{plt}\PY{o}{.}\PY{n}{show}\PY{p}{(}\PY{p}{)}
\end{Verbatim}


    
    \begin{verbatim}
<Figure size 1600x1000 with 1 Axes>
    \end{verbatim}

    
    \begin{Verbatim}[commandchars=\\\{\}]
{\color{incolor}In [{\color{incolor}4}]:} \PY{c+c1}{\PYZsh{}Sample plot points }
        \PY{k+kn}{import} \PY{n+nn}{numpy} \PY{k}{as} \PY{n+nn}{np}
        \PY{k+kn}{import} \PY{n+nn}{matplotlib}\PY{n+nn}{.}\PY{n+nn}{pyplot} \PY{k}{as} \PY{n+nn}{plt}
        \PY{o}{\PYZpc{}}\PY{k}{matplotlib} inline 
        
        \PY{n}{x}\PY{o}{=}\PY{p}{[}\PY{l+m+mi}{1}\PY{p}{,}\PY{l+m+mi}{2}\PY{p}{,}\PY{l+m+mi}{3}\PY{p}{,}\PY{l+m+mi}{4}\PY{p}{,}\PY{l+m+mi}{5}\PY{p}{]}
        \PY{n}{y}\PY{o}{=}\PY{p}{[}\PY{l+m+mi}{1}\PY{p}{,}\PY{l+m+mi}{2}\PY{p}{,}\PY{l+m+mi}{5}\PY{p}{,}\PY{l+m+mi}{8}\PY{p}{,}\PY{l+m+mi}{10}\PY{p}{]}
        \PY{n}{plt}\PY{o}{.}\PY{n}{plot}\PY{p}{(}\PY{n}{x}\PY{p}{,}\PY{n}{y}\PY{p}{,}\PY{l+s+s1}{\PYZsq{}}\PY{l+s+s1}{bo}\PY{l+s+s1}{\PYZsq{}}\PY{p}{,} \PY{n}{markersize}\PY{o}{=}\PY{l+m+mi}{10}\PY{p}{,}\PY{n}{alpha}\PY{o}{=}\PY{l+m+mf}{0.1}\PY{p}{)}
\end{Verbatim}


\begin{Verbatim}[commandchars=\\\{\}]
{\color{outcolor}Out[{\color{outcolor}4}]:} [<matplotlib.lines.Line2D at 0x116e09cf8>]
\end{Verbatim}
            
    \begin{center}
    \adjustimage{max size={0.9\linewidth}{0.9\paperheight}}{output_7_1.png}
    \end{center}
    { \hspace*{\fill} \\}
    
    \subsubsection{Problem 3 - Benchmarking Problem (25
Points)}\label{problem-3---benchmarking-problem-25-points}

Numpy has in-built universal functions that allow operations on
multidimentional arrays. For this problem, you will be asked to test the
performance of some of this functions.

For this, create an \texttt{NxN} matrix \texttt{A} and an array
\texttt{B} of size \texttt{N}, with random entries (use
\texttt{numpy.random}). Using \texttt{A} and \texttt{B} as inputs,
performing the following operations: (compare and time the performance
of sorting algorithms with and without numpy ufunc)

\begin{enumerate}
\def\labelenumi{\arabic{enumi}.}
\tightlist
\item
  Sorting \texttt{B} (you can use \texttt{numpy.sort}) for \texttt{N}
  ranging from 1000 to 25000 (take steps of 1000)
\item
  Computing Determinant of \texttt{A} ( you can use
  \texttt{numpy.linalg.det} to compute the determinant of the matrix)
  for \texttt{N} ranging from 100 to 800 (take steps of 50)
\item
  Matrix Multiplication of \texttt{A*A} ( use \texttt{numpy.matmul}) for
  \texttt{N} ranging from 100 to 1500
\end{enumerate}

Plot the operation times for Numpy functions vs regular python commands
for each operation.

The final output must include 3 plots comparing the performance of
universal functions in Numpy with similar analogs in python for each
operation. Provide a breif interpretation from the plots. (Feel free to
test around the range of \texttt{N} for this problem).

    \begin{Verbatim}[commandchars=\\\{\}]
{\color{incolor}In [{\color{incolor} }]:} \PY{c+c1}{\PYZsh{}Enter your solution here}
\end{Verbatim}



    % Add a bibliography block to the postdoc
    
    
    
    \end{document}
